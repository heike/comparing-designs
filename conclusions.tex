% !TEX root = comparison.tex
\section{Conclusions}~\label{conclusions}
Lineup charts allow us to assess and quantify the strength of a visual finding. In this paper we introduce the concept of  the power of a lineup, and show its use as a measure to evaluate competing designs. 
Power of a lineup can be modeled in a logistic regression model with subject-specific random effects, if observers were exposed and provided feedback to multiple lineups. 

We describe two MTurk studies evaluating different designs.


Could say something about adding a visual pleasure rating to the results, rate how attractive the display is on a scale of 1-5.

Familiarity with the design?

Assessing confidence leads to interesting discrepancies between perceived versus actual accuracy. In both examples, the only significant increases in confidence were not indicative of an increase in actual performance, in one case even the opposite happened. 

Discussion of demographics: both studies have highly educated participants; turk5 slightly more men. there's a gender * design interaction - did not include in the model write-up, though. too tired for today
turk4 has almost twice as many men.

cross-tabulation of age vs education gives odd results - 108 participants between 18-25 have a graduate degree?

