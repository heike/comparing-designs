% !TEX root = comparison.tex
\section{Conclusions}~\label{conclusions}
Lineup charts allow us to assess and quantify the strength of a visual finding. In this paper we introduce the concept of  the power of a lineup, and show its use as a measure to evaluate competing designs. 
Power of a lineup can be modeled in a logistic regression model with subject-specific random effects, if observers were exposed and provided feedback to multiple lineups. 

We describe two MTurk studies, where we use lineups to  evaluate competing designs. By varying the experimental conditions we are able to compare designs under slightly different  circumstances, which allows us to come to more general conclusions about weaknesses and strengths of one design over other designs.  While the MTurk results supported our choice of dotplots in the very small study that triggered our investigation, the results also brought the message of visual complexity home: showing much more information than necessary for a particular decision is detrimental to accuracy rather than supportive.

Assessing confidence leads to interesting discrepancies between perceived versus actual accuracy. In both examples, the only significant increases in confidence were not indicative of an increase in actual performance, in one case even the opposite occurred. 
Besides accuracy and speed, there is a variety of other aspects of the design, which could be collected in  the same framework of  MTurk studies, such as ease of use, familiarity with a particular type of design, or a visual pleasure rating of the lineups  (on a scale of one to five).




