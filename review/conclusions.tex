% !TEX root = comparison-review.tex
\section{Conclusions}~\label{conclusions}

%In this paper we have reported on results of using lineups for evaluating plot designs using a measure called power.  A logistic regression model with subject-specific random effects is used to obtain a model for estimating power, and account for individual subject abilities. 
%
%%Lineup charts allow us to assess and quantify the strength of a visual finding. In this paper we introduce the concept of  the power of a lineup, and show its use as a measure to evaluate competing designs. Power of a lineup can be modeled in a logistic regression model with subject-specific random effects, if observers were exposed and provided feedback to multiple lineups. 
%
%We describe two MTurk studies, where we use lineups to  evaluate competing designs. By varying the experimental conditions we are able to compare designs under slightly different circumstances, which allows us to come to more general conclusions about weaknesses and strengths of one design over other designs.  Cartesian charts are significantly more powerful than polar charts. The MTurk results also supported our choice of dotplots in the very small samples that triggered our investigation, but the results elucidated the visual complexity dilemma: showing much more information than necessary for a particular decision is detrimental to accuracy rather than supportive.
%
%Assessing the self-reported confidence leads to interesting discrepancies between perceived versus actual accuracy. In both examples, the only significant increases in confidence were not indicative of an increase in actual performance, in one case even the opposite occurred. 
%
%Besides accuracy and speed, there is a variety of other aspects of the design, which could be collected in  the same framework of  MTurk studies, such as ease of use, familiarity with a particular type of design, or a quantitative visual pleasure rating of the lineups.




In this paper we are investigating lineups for evaluating competing plot designs based on their power modeled by a logistic regression with subject-specific random intercepts. 
Lineups provide a powerful  tool for evaluating different designs in the framework of the data. We can show in two MTurk studies, that cartesian charts are significantly more powerful than polar charts. The  results also supported our choice of dotplots in the very small samples that triggered our investigation, but  elucidated the visual complexity dilemma: showing much more information than necessary for a particular decision is detrimental to accuracy rather than supportive. 

While we --theoretically-- can't show the same data plot to a single individual, it does make in practice a quite convincing tool to do so, e.g. in the informal setting of a classroom or in interactions with collaborators, when we do not intend to collect data for further investigation of a problem. 
Using the MTurk service gives us fast feedback -- both of the studies were completed within hours of their availability -- but comes at the price that we have to deal with the occasional `gamer'. For future studies we are planning on having a required entrance  test to try to avoid those. Besides accuracy and speed, there is a variety of other aspects of the design, which could be collected in  the same framework of  MTurk studies, such as ease of use, familiarity with a particular type of design, or a quantitative visual pleasure rating of the lineups.

Different visual abilities of individuals make up a significant amount of variability in lineup evaluations. Further investigations on how these relate to known  tests of cognitive skills, such as e.g. the paper folding tests.
